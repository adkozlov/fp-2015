\documentclass[12pt,a4paper]{article}

\usepackage[english,russian]{babel}
\usepackage[utf8]{inputenc}
\usepackage[T2A]{fontenc}
\usepackage{listings,comment,hyperref,listings,verbatim,amssymb}
\usepackage[nohead]{geometry}
\usepackage{pgfplots}

\begin{document}

\title{Домашнее задание №3}
\author{Андрей Козлов}
\date{\today}

\maketitle

\begin{enumerate}
\item {
	\begin{enumerate}
		\item $(\alpha \rightarrow \beta) \rightarrow \alpha \rightarrow \beta$
		\item $\alpha \rightarrow (\alpha \rightarrow \beta) \rightarrow \beta$
		\item $\alpha \rightarrow \alpha \rightarrow \alpha$
            \textcolor{red}{Это не наиболее общий тип}
		\item $(\alpha \rightarrow \beta) \rightarrow ((\alpha \rightarrow \beta) \rightarrow \alpha) \rightarrow \beta$
		\item {
			Терм не типизируется.\\
			Рассмотрим предтерм $(x y) x$, пусть он является термом. Тогда $\exists \Gamma, \sigma \colon \Gamma \vdash (x (y x)) \colon \sigma$.\\
			Тогда по лемме об инверсии правый подтерм $x$ имеет некий тип $\tau$, а левый подтерм $x y$ тип $\tau \rightarrow \sigma$, то есть $y \colon \alpha, x \colon \alpha \rightarrow \tau \rightarrow \sigma$. Таким образом, тип $\tau =  \alpha \rightarrow \tau \rightarrow \sigma$ является подвыражением себя, что невозможно в силу конечности типа.
		}
	\end{enumerate}
}
\item
\item
{
	\begin{enumerate}
		\item {
			\begin{lstlisting}[language=Haskell]
false :: Nat -> Bool -> Bool
false n _ = False

isZero :: Nat -> Bool -> Bool
isZero n _ = rec True false n

ge :: Nat -> Nat -> Bool
ge n m = rec True isZero (minus m n)
			\end{lstlisting}
		}
		\item {
			\begin{lstlisting}[language=Haskell]
multSucc :: Nat -> (Nat -> Nat)
multSucc n = mul (succ n)

fac :: Nat -> Nat
fac = rec 1 multSucc
			\end{lstlisting}
		}
		\item {
%			\begin{lstlisting}[language=Haskell]
%f :: (Nat -> a) -> Nat -> a
%f g 0 = g (suc 0)
%f g (suc n) = g (f g n)
%
%ack :: Nat -> Nat -> Nat
%ack 0 = suc
%ack (suc m) = f (ack m)
%			\end{lstlisting}
		}
	\end{enumerate}
}
\item {
	\begin{enumerate}
	\item {
		\begin{itemize}
			\item $Pair_{\sigma,\tau}$
			\item $\Gamma \vdash pair_{\sigma,\tau} \colon \sigma \rightarrow \tau \rightarrow Pair_{\sigma,\tau}$
			\item {
				\begin{itemize}
					\item $\Gamma \vdash fst_{\sigma} \colon Pair_{\sigma,\tau} \rightarrow \sigma$
					\item $\Gamma \vdash snd_{\tau} \colon Pair_{\sigma,\tau} \rightarrow \tau$
				\end{itemize}
			}
			\item {
				\begin{itemize}
					\item $\Gamma \vdash$ \texttt{fst (pair x y)} $\rightarrow$ \texttt{x}
					\item $\Gamma \vdash$ \texttt{snd (pair x y)} $\rightarrow$ \texttt{y}
				\end{itemize}
			}
		\end{itemize}
	}
	\item {
		\begin{itemize}
			\item $List_{\sigma}$
			\item {
				\begin{itemize}
					\item $\Gamma \vdash nil_{\sigma} \colon List_{\sigma}$
					\item $\Gamma \vdash cons_{\sigma} \colon \sigma \rightarrow List_{\sigma} \rightarrow List_{\sigma}$
				\end{itemize}
			}
			\item $\Gamma \vdash rec_{List_{\sigma}} \colon \alpha \rightarrow (\sigma \rightarrow List_{\sigma} \rightarrow \alpha \rightarrow \alpha) \rightarrow List_{\sigma} \rightarrow \alpha$
			\item {
				\begin{itemize}
					\item $\Gamma \vdash$ \texttt{rec} $_{List_{\sigma}}$ \texttt{n c nil} $\rightarrow$ \texttt{n}
					\item $\Gamma \vdash$ \texttt{rec} $_{List_{\sigma}}$ \texttt{n c (cons x xs)} $\rightarrow$ \texttt{c x xs (rec n c xs)}
				\end{itemize}
			}
		\end{itemize}
	}
	\end{enumerate}
}
\item
\lstinputlisting[language=Haskell]{sort.hs}
\end{enumerate}

\end{document}
